% !TeX encoding = UTF-8
% !TeX spellcheck = en_GB
% !TeX root = ../thesis.tex


\chapter{Introduction}
\label{chapter:introduction}

5G Networks have a growing relevance worldwide due to its improvements in speed, capacity and latency. A report stated 94 percent of respondents expect 5G growth to increase security and reliability concerns among 5G mobile operators, literature suggests that the potential threat of DDoS attacks has increased and is among the most serious attacks of all in 5G networks, since it prevents users from accessing network services \cite{huang2021trend}. Network slicing is one of the main features, that will bring 5G to the next level as a one-size-fits-all design is not suitable to handle its diverse and challenging requirements \cite{zhang2019overview}. But especially the performance and availability of network slices can be impacted by these types of attacks because they can share physical resources among each other. Therefore the interest in finding methods to manage these attacks is on the rise. A useful tool against this approach can be slice isolation, which is also an essential requirement for 5G networks \cite{sattar2019towards}.

To get a better understanding of the procedure of DDoS attacks and of potential measures against them, we want to focus on the following research questions:
\begin{addmargin}[20pt]{0pt}
    \begin{itemize}
        \item \textbf{RQ1: }\textit{What are the effects of DDoS attacks in a 5G network?}
        \item \textbf{RQ2: }\textit{What benefits does slicing have in case of DDoS attacks?}
    \end{itemize}
\end{addmargin}

For Background this paper will go over 5G networks in general, network slicing and the upcoming relevance of DDos attacks in 5G networks. The main focus will be on the implementation of the simulation using ns-3 with 5G-Lena and the data gathering process. Lastly the collected data will be visualized, evaluated and discussed.