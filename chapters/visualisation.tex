% !TeX encoding = UTF-8
% !TeX spellcheck = en_GB
% !TeX root = ../thesis.tex

\NewDocumentCommand{\codeword}{v}
	%\textcolor{blue}{
}

\chapter{Tracing and Visualization}
\label{chapter:tracing}
In this section we describe how we retrieve tracing data from the simulation, output the tracing data to files in a specific format, parse this format with other programs/languages for further analysis and how we visualize the simulation using external software.

\section{Basic Metrics of the Packet Flows}
The main source of our data are the Flow Summary Files. They are created manually by our function \codeword{traceMetricsSummary} in the tracing file.
Using the \codeword{ns3::FlowMonitor} we collect the following data:
\begin{itemize}
	\item Total number of transmitted packets for the flow (Tx Packets)
	\item Total number of transmitted bytes for the flow (Tx Bytes)
	\item Rate of transmitted bits for the flow (TxOffered)
	\item Total number of received bytes for the flow (Rx Bytes)
	\item Rate of recieved bits for the flow (Throughput)
	\item Mean delay per recieved packet for the flow (Mean delay)
	\item Mean jitter per recieved packet for the flow (Mean jitter)
\end{itemize}

With the function \codeword{traceNodeContainerIPs} we additionally store which nodes -- identified by their IP address -- are assigned to which slice (and which nodes are the base stations) during the simulation.

Scenario 1 results in a flow file that looks like this:

\begin{verbatim}
Flow 1 (1.0.0.2:49153 -> 7.0.0.6:1234) proto UDP
  Tx Packets: 6000
  Tx Bytes:   768000
  TxOffered:  10.240000 Mbps
  Rx Bytes:   767104
  Throughput: 10.228053 Mbps
  Mean delay:  0.786401 ms
  Mean jitter:  0.119959 ms
  Rx Packets: 5993
Flow 2 (1.0.0.2:49154 -> 7.0.0.4:1235) proto UDP
  Tx Packets: 60
  Tx Bytes:   7680
  TxOffered:  0.102400 Mbps
  Rx Bytes:   7680
  Throughput: 0.102400 Mbps
  Mean delay:  1.756546 ms
  Mean jitter:  0.049405 ms
  Rx Packets: 60
Flow 3 (1.0.0.2:49155 -> 7.0.0.5:1235) proto UDP
  Tx Packets: 60
  Tx Bytes:   7680
  TxOffered:  0.102400 Mbps
  Rx Bytes:   7680
  Throughput: 0.102400 Mbps
  Mean delay:  1.827379 ms
  Mean jitter:  0.048810 ms
  Rx Packets: 60
Flow 4 (1.0.0.2:49156 -> 7.0.0.2:1236) proto UDP
  Tx Packets: 6000
  Tx Bytes:   768000
  TxOffered:  10.240000 Mbps
  Rx Bytes:   764160
  Throughput: 10.188800 Mbps
  Mean delay:  2.794291 ms
  Mean jitter:  0.179825 ms
  Rx Packets: 5970
Flow 5 (1.0.0.2:49157 -> 7.0.0.3:1236) proto UDP
  Tx Packets: 6000
  Tx Bytes:   768000
  TxOffered:  10.240000 Mbps
  Rx Bytes:   764160
  Throughput: 10.188800 Mbps
  Mean delay:  2.867154 ms
  Mean jitter:  0.179801 ms
  Rx Packets: 5970


  Mean flow throughput: 6.162091
  Mean flow delay: 2.006354

Nodes: 
  EMBB Slice: 
    7.0.0.2
    7.0.0.3
  MMTC Slice: 
    7.0.0.4
    7.0.0.5
  URLLC Slice: 
    7.0.0.6
  Ground Stations: 
    10.0.0.5
\end{verbatim}

\section{Trace Data of all Device Operations with Packets}
Using the PointToPointHelper and the AsciiTraceHelper of ns-3, we create a .tr file that contains one trace event per line. 
The first character in that line describes the operation that happened on the device queue and can be one of these:
\begin{itemize}
    \item \texttt{+} enqueue
    \item \texttt{-} dequeue
    \item \texttt{d} packet dropped
    \item \texttt{r} packet received
\end{itemize}
The next value, separated with a space, is the current simulation time in seconds.
More details about this format can be found in the ns-3 tutorials. \footnote[11]{\url{https://www.nsnam.org/docs/release/3.9/tutorial/tutorial_23.html##:~:text=5.3.1.1}}

\section{Tracing and 5G New Radio}
\subsection{About ns-3 Tracing and Callbacks}
To build an own solution for tracing NR traffic, an in-depth analysis of ns-3s trace system was necessary.

The interface ns-3 provides is the PointToPointHelpers EnableAsciiAll-method. Using this method tracing data written to the specified output stream can be obtained. Internally the PointToPointHelper uses other helpers such as the WimaxHelpers EnableAsciiInternal. These helpers have callback methods (trace sinks) on which our output stream gets bound to, so the callbacks write a line of tracing data once called. The callbacks are then connected to the corresponding ns-3 objects. 

\subsection{Tracing of 5G NR traffic}
\label{subsection:tracing-5g-nr}
The previously described method does not yield any results for the traffic between the UE devices and the gNodeBs over new radio. To overcome this issue, an implementation of tracing with callbacks was done for the NR UE net devices. 

By overwriting the \texttt{Receive(Ptr<Packet> p)} method of NrNetDevice in the NrUeNetDevice implementation, access to the packets received by a NrUeNetDevice can be gained.

To provide a clean design and interface to use this data, trace sources and callbacks are added to the NrUeNetDevice. With these you can use the events and their data analogous to the rest of the ns3 devices and objects.

In our simulation the function, that gets called once an MacRx action in a NrUeNetDevice happens, is \texttt{nrTraceCallbackRxTr} in the tracing.cc file. It accepts an OutputStreamWrapper and context as bound parameters (specified on creation of the callback) and the received packet as argument. Using this, an output in the same format as in the .tr files can be created and written to a file.

By comparing the output we gain through this callback with the data present in the .tr files, we can see that we obtain new data that is not already present in the default .tr files. While the same packets of course appear in the .tr file (as they're transmitted through different devices, including some not-NR ones), their simulation time and context (containing the node and device number) does not match with the context and time the packets are received at the NR devices. 

While this works great for the receive-event, it does not for send-events, as the methods handling the sending of data are never called in the NrUeNetDevices.

\textit{Conclusion:} In general the same trace system ns-3 uses everywhere can be applied to NR devices provided by the 5G-Lena module, as demonstrated successfully on the receive-event of NrUeNetDevices. Implementing the send, drop and other actions would need a deeper understanding of ns-3 and 5G-Lena and therefore more time.

\section{Visualization in NetAnim}
\subsection{Generating NetAnim XML Data}
We already described the installation process and the usage of NetAnim in \ref{subsection:InstallNetAnim}. Here we focus on how NetAnims AnimationInterface can be called from our own source code and how we can track, customize and visualize the traffic of our simulation.

First the AnimationInterface is instantiated and the file where the XML output will be written to is specified.

In the next step we define the mobility model of our nodes. Because we want to arrange them in a static grid, we install the ns3::ConstantPositionMobilityModel on all node containers. Then a function in our tracing.cc file is called which takes care of positioning the nodes of each container in a line.
\subsection{Adding our own data to the visualization}
While NetAnim collects the basic data itself, we asynchronously collect data with our own trace sink that we connect to the receive-event of NrUeNetDevices (see~\ref{subsection:tracing-5g-nr}). The callback we implemented for this uses the bound information about the device as well as the dynamic packet information and the current simulation time to generate a new XML p-tag for each receive-event. This data is written to a temporary XML file and combined with the XML data generated by NetAnim after the simulation ends. The additional XML data generated by us also contains links (XML link-tags NetAnim uses to display the lines between nodes) between the NR UE nodes and their gNodeB. The combining of both files is done in a separate program to ensure NetAnim finished writing to the XML file (as it does this asynchronous after the simulation).






